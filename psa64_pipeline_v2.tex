\documentclass[onecolumn]{emulateapj}
%\documentclass[preprint]{aastex6}
%\documentclass{article}
\usepackage{affils}
\usepackage{url}
\usepackage{multirow}
\usepackage{amsmath}
\usepackage[dvipsnames]{xcolor}
\usepackage{enumerate}
\usepackage{listings}
\usepackage{rotating,afterpage}

\lstset{language=bash,numbers=left,
		numberstyle=\tiny ,
		stepnumber=1,
		numbersep=8pt,
		basicstyle=\footnotesize, 
		breaklines=true,
		showspaces=false,
		showstringspaces=false,
		frame=tblr,
		backgroundcolor=\color{lightgray} }
%\citestyle{aa} 
%\bibliographystyl{apj_w_etal}

\newcommand{\etal}{{et al.\/}}
\newcommand{\Prob}{\mathtt{P}}
\newcommand{\logL}{\log\mathcal{L}}
\newcommand{\unit}[1]{\footnotesize #1}
\newcommand{\PAPER}{\mathrm{PAPER}}
\newcommand{\hMpci}{h\ {\rm Mpc}^{-1}}

%%define graphics path to search for images
%\graphicspath{{./data/}}

\newcommand{\Nconf}{31}
\newcommand{\Nsrc}{32}
\definecolor{orange}{RGB}{255,127,0}

\tabletypesize{\scriptsize}

	% End definitions

%\slugcomment{DRAFT: \today}

%\shortauthors{Kolopanis}


\begin{document}

\title{Running the PSA64 Power Spectrum Pipeline}

\DeclareAffil{asu}{Department of Physics, Arizona State University, Tempe, AZ 85203}
\DeclareAffil{sese}{School of Earth and Space Exploration, Arizona State University, Tempe, AZ 85203}


\affilauthorlist{
	Matthew Kolopanis%\affils{asu,sese}
}
	
	
	
\maketitle	
Here I present an overview of the PAPER power spectrum pipeline used in creating the A15 and K16 power spectra. This memo aims to clarify the multiple steps of the pipeline including: setting up your configuration file, selecting data, and running signal loss scripts. I will conclude with a brief discussion on using the psa128 pipeline.
\newline

\section{PSA64 Power Spectrum Pipeline}{
The psa64 pipeline relies on two files found in capo/pspec\_pipeline/:


\begin{enumerate}
\item the configuration file: pspec\_psa128.cfg
\item the bash power script: mk\_psa128\_pspec.sh
\end{enumerate}
\subsection{Configuration File} 
{
The configuration file is used to define key parameters used in the making of the power spectrum. Table \ref{tab:parms} provides a list of all parameters, their descriptions and some examples.


\begin{sidewaystable}[h]
\centering
\begin{tabular}{c | c}
Parameter & Description  \\
\hline 
     &  Title of output directory, figures and save files (npz)\\
PREFIX & Recommend adding the Date for easy logging \\
     & e.g Mar30\_optimal\_frf \\ \hline
	%& I like to use something descriptive about the power spectrum too \\ \hline
pols & 	 The polarization you wish to analyze \\
	 & options like I, XX, YY, XY \\ \hline
     & the antenna separations to be used sep0,1 sep1,1 sep-1,1 etc\\
seps & your data directory should have these as subdirectories\\ 
	 & the pipeline looks for these directories to collect data\\ \hline
      & channel ranges to be analyzed \\
chans & can be singe channel range: 95\_115 \\
      & or list of ranges: 30\_50 51\_71 78\_98  103\_123 127\_147  \\ \hline     
	 & Antennas to analyze in pspec \\
ANTS	 & can be set to any AIPY compliant antenna value\\
	 & use 'cross' as default \\ \hline
LST  & The RA (LST) of nights you wish to analyze \\
 	& expects input to be a range: .4\_9 \\
 \hline
NBOOT & Number of bootstraps to perform \\ \hline
FILEAPPELLATION & file suffix \\
			    & e.g. uvGAL uvG uvGL uvGLS uvGLAS etc \\ \hline
EVEN\_GLOB & A bash glob of even files \\
		  & e.g. lst.*242.[3456]* \\ \hline
ODD\_GLOB & A bash glob of odd files \\
		  & e.g. lst.*243.[3456]* \\ \hline
	& signal loss correction factor \\
covs & must have same length as chans keyword\\
	 & e.g '1.02 1.2 1.30 1.37 1.24'\\ 
	  & no input ('') defaults to 1.0 for all chans 
\end{tabular}
\begin{tabular}{c | c}
Parameter & Description  \\ \hline 
      & amplitude of noise to inject \\
 noise	  & must have same length as chan\\
 	  & or null for no noise\\ \hline
FILTER\_NOISE & Boolean \\
	  & True to fringe rate filter injected noise\\ \hline
NOISE\_ONLY & Boolean \\
	  & True to replace data with noise when injecting \\ \hline
USE\_pI & Boolean \\
	   & True to use pI in pspec\_cov\_v???.py \\ \hline
SCRIPTSDIR &  Path to directory where scripts are located\\ \hline
cal & the name of the cal file to use \\ 
	&  must be in your path or your python sys.path \\ \hline
PWD & pwd command given to script for logging \\ \hline
EVEN\_DATAPATH & path to even nights \\ \hline
ODD\_DATAPATH & path to odd nights\\ \hline 
 	& Window function to multiply data by \\ \hline
WINDOW 	& e.g. Blackman-Harris, Kaiser3\\
	& current default is none\\ \hline
 PLOT & Boolean to plot to X-window \\
 	& True or False \\ \hline
COV  & Boolean to turn On and Off covariance \\
	& True or False\\ \hline
BOOT & Boolean to turn on and off bootstrapping \\
	& True or False \\ \hline
KPKPLOT & Boolean to turn on and off plotting\\
		& True or False
\end{tabular}
	
\caption{\label{tab:parms}List of Parameters used in configuration file for use with mk\_psa128\_pspec.sh}
\end{sidewaystable}


Inside of the configuration file, either the EVEN\_GLOB and ODD\_GLOB keywords or the LST keyword should be set. The globs should be use to manually pick data files to use, these keywords take preference over the lst\_select script. 
Inside your configuration file, all parameters must be given in the form:

\begin{figure}[h!]
\centering
\begin{lstlisting}[numbers=none]
 export PARAMETER='STRING OF PARAMETER VALUE'
\end{lstlisting}

\end{figure}
 
\clearpage
A sample config file would look like this:


\begin{tabular}{c}
\begin{lstlisting}[name='pspec_psa128.cfg']
#this file configures a pspec run
# run with mk_pspec.sh <this file>


export PREFIX='Apr22_optimal_frf_test_noise_only_31Jy'

#chans=`python -c "print ' '.join(['%d_%d'%(i,i+39) for i in range(10,150,1)])"`
export pols='I'
export seps='sep0,1 sep1,1 sep-1,1'
#export chans='30_50 51_71 78_98 95_115 103_123 127_147'
export chans='95_115'
export ANTS='cross'
#export chans='95_115'
#export RA="1:01_9:00"
export NBOOT=100
export FILEAPPELLATION='uvGAL'

## use EVEN_GLOB and ODD_GLOB to manually select data
## script will use manaul glob over lst_select
export EVEN_GLOB='lst.*242.[3456]*'
export ODD_GLOB='lst.*243.[3456]*'
export LST="-.1_8.75"

#signal loss correction factor
#export covs='1.20 1.19 1.23 1.22 1.24 1.28'
#export covs='1.62 1.35 1.30 1.30 1.28 1.35'
export covs='1.26'

##amplitude of injected noise in Jansky
export noise='30'
##Fringe Rate Filter noise before injection
export FILTER_NOISE=True
#instead of injecting noise, override data with noise
export NOISE_ONLY=True

#replace pC with pI in pspec_cov_boot_v???.py
export USE_pI=False

#DATAPATH=fringe_hor_v006
export SCRIPTSDIR=~/src/capo/pspec_pipeline
export cal='psa6240_v003'
export PWD=`pwd`
export EVEN_DATAPATH="${PWD}/lstbin_psa64_ali_optimal/even"
export ODD_DATAPATH="${PWD}/lstbin_psa64_ali_optimal/odd"
export WINDOW='none'

#to separately run scripts
export PLOT=False #to plot things in COV and BOOT scripts
export COV=True
export BOOT=True
export KPKPLOT=True


\end{lstlisting}
\end{tabular}
}
\subsection{Running the Bash Script}{
Now we have our configuration file set, we need to get ready to run the power spectrum. With how the sample config file is set, we should compute the power spectrum in the directory which contains the data folder (lstbin\_psa64\_ali\_nofrf). Once in the correct directory, we can run the bash script with the command:

\begin{figure}[h!]
\centering
\begin{lstlisting}[numbers=none]
user@folio /path/to/script/mk_psa128_pspec.sh /path/to/config/pspec_psa128.cfg
\end{lstlisting}
\end{figure}


if we are running this directly from the capo/pspec\_pipeline directory it will look like:
\begin{figure}[h!]
\centering
\begin{lstlisting}[numbers=none]
user@folio /path/to/capo/pspec_pipeline/mk_psa128_pspec.sh /path/to/capo/pspec_pipeline/pspec_psa128.cfg
\end{lstlisting}
\end{figure}
}
}


%\begin{deluxetable}{c | l }
%\centering
%\tablecaption{List of Parameters used in signal loss configuration file for use with make\_sigloss.sh}
%\tablecolumns{2}
%\tablehead{
%\colhead{Parameter} &
%\colhead{Description}
%}
%\startdata
\begin{table}[ht]
\centering
\begin{tabular}{c | c}
Parameter & Description \\ \hline \hline
     &  Title of your output director, figures and save files (npz)  \tabularnewline
PREFIX & Recommend adding the Date in your prefix for easy logging \tabularnewline
     & e.g Mar30\_sigloss\_optimal\_frf \tabularnewline \hline
PSPEC & This is the name of the power spectrum we computed before\tabularnewline
	 & probably the same as PREFIX from your pspec\_psa128.cfg file\tabularnewline \hline
 PLOT & N/A\tabularnewline \hline
COV  &  N/A \tabularnewline \hline
BOOT &  N/A \tabularnewline \hline
KPKPLOT & N/A\tabularnewline
\end{tabular}
\caption{\label{tab:sig_parms}List of Parameters used in signal loss configuration file for use with make\_sigloss.sh}
\end{table}
%\enddata
%\label{tab:sig_parms}
%\end{deluxetable}




\section{Signal Loss}
{
The signal loss calculation also relies on two main files:
\begin{enumerate}
\item the configuration file: sigloss\_psa64.cfg
\item the bash power script: make\_sigloss.sh
\end{enumerate}

The signal loss configuration file has a few differences outlined in Table \ref{tab:sig_parms}




A sample signal loss config file looks like: 



\begin{lstlisting}[name='sigloss_psa128.cfg']
#this file configures a pspec run
# run with mk_pspec.sh <this file>


export PREFIX='Mar30_sigloss_optimal_frf'
export PSPEC='Mar30_optimal_frf'

#chans=`python -c "print ' '.join(['%d_%d'%(i,i+39) for i in range(10,150,1)])"`
export pols='I'
#export seps='sep0,1 sep-1,1 sep1,1'
export seps='sep0,1'
export chans='30_50 51_71 78_98 95_115 103_123 127_147'
ANTS='cross'
#export chans='95_115'
#export RA="1:01_9:00"
export NBOOT=40
export FILEAPPELLATION='uvGAL'

## use EVEN_GLOB and ODD_GLOB to manually select data
## script will use manaul glob over lst_select
export EVEN_GLOB='lst.*242.[3456]*'
export ODD_GLOB='lst.*243.[3456]*'
export LST="-.1_8.75"

#DATAPATH=fringe_hor_v006
export SCRIPTSDIR=~/capo/pspec_pipeline
export cal='psa6240_v003'
export PWD=`pwd`
export EVEN_DATAPATH="${PWD}/lstbin_psa64_ali_nofrf/even"
export ODD_DATAPATH="${PWD}/lstbin_psa64_ali_nofrf/odd"
export WINDOW='none'
\end{lstlisting}
\ \newline

Running signal loss is very similar to the power spectrum:

\begin{figure}[h!]
\begin{lstlisting}[numbers=none]
user@folio /path/to/script/make_sigloss.sh /path/to/config/sigloss_psa128.cfg
\end{lstlisting}
\end{figure}

if we are running this directly from the capo/pspec\_pipeline directory it will look like:

\begin{figure}[h!]
\begin{lstlisting}[numbers=none]
user@folio /path/to/capo/pspec_pipeline/make_sigloss.sh /path/to/capo/pspec_pipeline/sigloss_psa128.cfg
\end{lstlisting}
\end{figure}

The signal loss script will output the maximum signal loss correction factors it calculates inside of the log file for each channel/polarization combination. These correction factors can be put back into the 'covs' keyword of the power spectrum configuration file to correct the output power spectrum.
}


\end{document}
