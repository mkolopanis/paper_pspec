\documentclass[onecolumn]{emulateapj}
%\documentclass[preprint]{aastex6}
%\documentclass{article}
\usepackage{affils}
\usepackage{url}
\usepackage{multirow}
\usepackage{amsmath}
\usepackage[dvipsnames]{xcolor}
\usepackage{enumerate}
\usepackage{listings}
\usepackage{rotating,afterpage}


\lstset{language=bash,numbers=left,
		numberstyle=\tiny ,
		stepnumber=1,
		numbersep=8pt,
		basicstyle=\footnotesize, 
		breaklines=true,
		showspaces=false,
		showstringspaces=false,
		frame=tblr,
		backgroundcolor=\color{lightgray} }
%\citestyle{aa} 
%\bibliographystyl{apj_w_etal}

\newcommand{\etal}{{et al.\/}}
\newcommand{\Prob}{\mathtt{P}}
\newcommand{\logL}{\log\mathcal{L}}
\newcommand{\unit}[1]{\footnotesize #1}
\newcommand{\PAPER}{\mathrm{PAPER}}
\newcommand{\hMpci}{h\ {\rm Mpc}^{-1}}

%%define graphics path to search for images
%\graphicspath{{./data/}}

\newcommand{\Nconf}{31}
\newcommand{\Nsrc}{32}
\definecolor{orange}{RGB}{255,127,0}

\tabletypesize{\scriptsize}

	% End definitions

%\slugcomment{DRAFT: \today}

%\shortauthors{Kolopanis}


\begin{document}

\title{Running the PAPER Power Spectrum Pipeline}

\DeclareAffil{asu}{Department of Physics, Arizona State University, Tempe, AZ 85203}
\DeclareAffil{sese}{School of Earth and Space Exploration, Arizona State University, Tempe, AZ 85203}


\affilauthorlist{
	Matthew Kolopanis%\affils{asu,sese}
}
	
	
	
\maketitle	
Here I present an overview of the PAPER power spectrum pipeline used in creating the A15 and K17 power spectra. This memo aims to clarify the multiple steps of the pipeline including including the use of the sigloss calculations in power spectrum estimation
\newline

\section{PAPER Power Spectrum Pipeline}{
The psa64 pipeline relies on four files found in capo/pspec\_pipeline/:


\begin{enumerate}
\item the bash script: sigloss\_bash\_v2.sh
\item the main pspec and sigloss script: sigloss\_sim.py
\item the boot strapping script: boots\_to\_pspec.py
\item Upper limit estimator: sigloss\_limits.py
\end{enumerate}

\subsection{sigloss\_bash\_v2}{
The bash script runs the other three parts of the power spectrum pipeline and requires some parameters be set to properly estimate the power spectrum.
\begin{table}[h]
\centering
\normalsize
\begin{tabular}{l | c}
Parameter & Description \\ \hline
PREFIX & path pointing to data folder\\
EVEN\_FILES & path and glob of even files\\
ODD\_FILES & path and glob of odd files\\
WD & the PWD command\\
noise & boolean used to make noise simulations of pspec \\
boot & number of bootstraps \\
chans & list of channel ranges in capo chan option format (e.g. 95\_115) \\
rmbls & list of baselines to remove in i,j pairs (e.g. 0\_44) \\
t\_eff & the effective integration time, in number of time bins \\
bl\_length & baseline length in meters \\
SEP & Separation type (e.g. 0,1)
\end{tabular}
\end{table}

here's what a sample bash configuration will look like:
\begin{figure}[h]
\centering
\begin{lstlisting}[firstnumber=0,name=sigloss_bash_v2.sh, caption={sigloss\_bash\_v2.sh},
fontadjust,linewidth=15cm,xleftmargin=.2\textwidth]
PREFIX='../../../../exp_vs_inttime/lstbin_psa64_data_optimal'
EVEN_FILES=${PREFIX}'/even/sep0,1/lst*242.[3456]*.uvGAL'
ODD_FILES=${PREFIX}'/odd/sep0,1/lst*243.[3456]*.uvGAL'
WD=$PWD #get the sworking directory
noise=''
boot=60
chans='95_115'
rmbls='15_16,0_26,0_44,16_62,3_10,3_25'
t_eff=69
bl_length=30
SEP='0,1'
\end{lstlisting}
\end{figure}

The remainder of the bash script executes the other files
with the configurations set above.
One additional line in the script is used to configure the 
injections used in the signal loss interactions:
\begin{figure}[h]
\centering
\begin{lstlisting}[firstnumber=40,name=sigloss_bash_v2.sh,
caption={sigloss\_bash\_v2.sh}]
    for inject in `python -c "import numpy; print ' '.join(map(str, numpy.logspace(-2,3,30)))"` ; do
\end{lstlisting}
\end{figure}

Changing the logspace beginning, end and number of points
will affect the number and range of injections.
}


\subsection{sigloss\_sim}{
The main power spectrum and signal loss estimation script.
This script has a few parameters still hard-coded in unfortunately.

\begin{table}[h]
\centering
\normalsize
\begin{tabular}{l | c}
Parameter & Description \\ \hline
LST\_STATS & Boolean to collected metadata from lst binning\\
DELAY & Boolean to take delay transform\\
NGPS & Number of groups used in bootstrapping\\
cov\_reg\_level & scaling factor of identity to add to covariance before inversion\\
\end{tabular}
\end{table}
other parameters previously set with this script are now
passed in via the command line option parser. 

An example of the configuration inside sigloss\_sim.py looks like: 

\begin{figure}[h]
\centering
\begin{lstlisting}[firstnumber=36,name=sigloss_sim.py,fontadjust,
linewidth=12cm,xleftmargin=.3\textwidth,caption={sigloss\_sim.py}]
random.seed(0)
POL = opts.pol  # 'I'
LST_STATS = False
DELAY = False
NGPS = 5
INJECT_SIG = opts.inject_sig
PLOT = opts.plot
cov_reg_level = 1e-10
\end{lstlisting}
\end{figure}


}


\subsection{boot\_to\_pspec}{
Boots to pspec takes output bootstrapped files for each injection
level and performs the second round of bootstraps.
It used the t\_eff parameter to construct a waterfall with only the 
effective number of free parameters from bootstraps of all
injections. Each time in the new waterfall is chosen randomly from a 
bootstrap file. This process is completed 100 times, then averaged
and the standard deviation is taken to estimate the power 
spectrum and its uncertainties.


This script also computes the folded power spectrum, resulting in
the only positive k-values for use in computing $\Delta^{2}$.
}


\subsection{sigloss\_limits}{
Sigloss limits uses the outputs of boots to pspec to compute 
the signal loss corrected power spectrum evaluating 
the integral to find the probability each injected power spectrum
value is greater than the 1-sigma upper limits of the data power
spectrum in each k-bin.
}

\end{document}
